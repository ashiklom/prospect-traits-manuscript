\section{Discussion}

\subsection{Estimating traits through PROSPECT inversion}

Establishing general, species- and site-independent relationships between leaf functional traits and optical properties is challenging.
My results suggest that Bayesian PROSPECT inversion is a promising technique for achieving this objective.
Averaged across the entire dataset, spectral estimates of traits were able to capture 50 to 75\% of the variability in the true values of the traits (Figure~\ref{fig:project_validation_summary}).
In addition, spectral trait estimates were able to not only identify but also ascribe a physiological mechanism to intra-specific variability associated with long-term acclimation and acute stress responses in both natural and experimental settings (Figure~\ref{fig:treatment_summary}).
By comparing retrieval accuracy across different versions, my results also reaffirm the value of recent improvements to PROSPECT\@.
In particular, the successive additions of carotenoid~\cite{feret_2008_prospect} and anthocyanin~\cite{feret_2017_prospectd} pigments significantly increased accuracy of chlorophyll retrievals in the phenological dataset of Yang et al.~(2016) (Figures~\ref{fig:prospect_D_validation} and~\ref{fig:project_validation_summary}), which points to the importance of modeling non-photosynthetic pigments in leaves sampled early or late in the growing season.   % SI Figure \ref{fig:validation_cab}

That being said, the large scale validation demonstrated here reveals enduring challenges and development opportunities for modeling leaf optical properties and retrieving leaf traits from spectra.
The particularly poor inversion accuracy for grasses (Figures~\ref{fig:prospect_D_validation} and~\ref{fig:prospect_D_r2}) suggests that the biochemistry and morphology of grasses do not fit the assumptions of PROSPECT\@.
Perhaps lower-hanging fruit is investigation of places where inversion estimates did well at capturing variability in traits, but were additively or multiplicatively biased.
For example, estimates of chlorophyll and carotenoid content in herbs from two dramatically different environments and species (greenhouse-grown soybean and tundra vegetation) showed virtually the same multiplicative bias and accuracy (Figure~\ref{fig:prospect_D_validation}).
One possible culprit for this bias is the chlorophyll $a$--$b$ ratio, which varies within and across species~\cite{kurahotta_1987_relationship,kitajima_2003_increases} but is fixed in the current version of PROSPECT\@.
Fortunately, a recently released update to PROSPECT succeeded in modeling chlorophyll $a$ and $b$ independently~\cite{zhang_2017_extended}.
This advance continues an ongoing positive trend in improving the detail with which PROSPECT models leaf optical properties.
 
One conclusion of Chapter 2 was that the use of physically-based absorption coefficients, such as that for leaf water content, is important for accurate trait retrievals using physically-based radiative transfer models.
Results from the wider range of species and projects in this chapter challenge this notion.
Retrievals of leaf water content and total chlorophyll concentration had comparable overall $R^2$ values (Figure~\ref{fig:project_validation_summary}).
However, leaf water content retrievals exhibited clear and significant project-specific biases, especially at high values (Figure~\ref{fig:prospect_D_validation})
Meanwhile, chlorophyll content retrieval was more consistently accurate across its entire range, even for needleleaved species (in the Di Vittorio dataset) (Figure~\ref{fig:prospect_D_validation}) that poorly fit the parallel-plane assumptions of the PROSPECT model~\cite{allen_1969_interaction,jacquemoud_1990_prospect} and, more importantly, despite the fact that the chlorophyll absorption coefficients for PROSPECT are calibrated only against the ANGERS dataset, which does not have any conifers~\cite{feret_2008_prospect,feret_2017_prospectd}.
Project-specific calibration has been shown to further improve the results of PROSPECT inversion~\cite{li_2013_retrieval}, which suggests that re-calibration of PROSPECT absorption coefficients against a wider range of species and environmental conditions (such as those used here) could lead to significant improvements in PROSPECT performance.
Ongoing efforts to curate and make publicly available spectral observations, such as the ECOSIS project (ecosis.org), significantly aid such efforts.

\subsection{Variation in optical traits}

Optical traits showed substantial variability both within and across species.
The extent of intra-specific variability in optical traits was substantial---from 30\% for leaf structure and dry matter content to nearly 50\% for pigment concentrations (Figure~\ref{fig:within_vs_across})---and fell comfortably in the range of intraspecific trait variability reported in other studies for similar traits~\cite{messier_how_2010,albert_multi-trait_2010}.
Following the definition of McGill et al.~(2006) that a useful ``trait'' is one that varies more across than within species, all six of the traits examined in this study technically qualify as ``traits'', but pigment concentrations only barely. \nocite{mcgill_2006_rebuilding}
In addition, the interspecific variability is poorly explained by species attributes typically used to define plant functional types (e.g.\ for dynamic vegetation models)---taken together, species-specific attributes were able to explain at most around 30\% of interspecific variability (Figure~\ref{fig:across_species_anova}).
This result adds to the emerging body of literature on the limited ability of discrete plant functional types with fixed traits to effectively capture variability in plant and ecosystem function~\cite{vanbodegom2012_beyond,vanbodegom2014_fully,verheijen2015_variation,clark_why_2016}.

Some of the intraspecific variability in optical traits was not random, but rather suggested a systematic plastic response to biotic and abiotic stressors (Figures~\ref{fig:treatment_summary} and~\ref{fig:trait_phenology}).
For example, the observed increase in leaf dry matter content with decreasing temperature and increasing precipitation both agree with the meta-analysis of leaf mass per area by Poorter et al.~(2009). \nocite{poorter_2009_causes}
On the other hand, the absence of significant trends in pigment and water contents with respect to site temperature are likely because these traits respond more rapidly to environmental conditions, which is supported by their relatively higher fraction of intra-specific variability (Figure~\ref{fig:within_vs_across}).
This idea is further supported by the fact that pigment concentrations, but not leaf structure or dry matter content, responded significantly to within-season temperature fluctuations in the Barnes et al.~(2017) \nocite{barnes_2017_beyond} dataset and to aphid pressure in the soybean aphid dataset (Figure~\ref{fig:treatment_summary}).
This positive response of chlorophyll concentration to aphid pressure is surprising.
Alves et al.~(2015) \nocite{alves_2015_soybean} found significant effects of aphid infestation on soybean near-infrared reflectance and NDVI but no effect of on chlorophyll content.
Meanwhile, Luo et al.~(2012) \nocite{luo_2012_evaluation} found that wheat aphid infestation increased wheat leaf reflectance across the visible and near-infrared range, consistent with reduced pigment concentrations.
This result is unlikely to be caused by inaccurate PROSPECT estimates of pigment concentrations because inversion accuracy of both chlorophyll and carotenoids for this dataset was among the highest in this study (Figures~\ref{fig:project_validation_summary} and~\ref{fig:prospect_D_validation}).

I observed statistically significant differences in leaf morphology and biochemistry between sunlit and shaded leaves.
Chlorophyll content was significantly higher in shade leaves compared to sun leaves, which supports established theory that allocation of resources to light absorption relative to other photosynthetic functions (e.g.\ carbon fixation) increases with decreasing irradiance~\cite{hikosaka_1995_model}.
At the same time, the reduced leaf dry matter content and mesophyll structure in shade leaves agrees with established understanding of the relationship between leaf mass per area and irradiance~\cite{poorter_2009_causes}.
However, the lack of a significant shade effect on carotenoid content and the positive effect on anthocyanins are surprising, given the current understanding of the photoprotective role of these pigments~\cite{young_1991_photoprotective,steyn_2002_anthocyanins}.
One explanation for the lack of a shade effect on carotenoids is that the response is non-linear, as has been shown in treatments with more finely varied light levels~\cite{sonobe_2017_estimating}.
An alternative, simpler explanation may be inaccuracy in retrievals related to the relative coarseness with which pigments are currently treated by PROSPECT (as discussed above).

Another source of intraspecific variability in optical traits explored in this study was phenology (Figure~\ref{fig:trait_phenology}).
PROSPECT inversion was able to successfully capture the phenological progression of chlorophyll and carotenoid contents as they increase early in the growing season and decline in the fall~\cite{yang_2014_beyond,yang_2016_seasonal}.
However, the results for leaf mass per area disagreed with the direct measurements in several important ways.
First, contrary to direct observations at this site~\cite{yang_2016_seasonal} and to expectations based on literature survey~\cite{poorter_2009_causes}, my estimates for leaf mass per area were consistently lower in sunlit than shaded leaves.
Second, while direct observations show that leaf mass per area generally increases early in the growing season up to leaf maturity and then remains effectively constant until leaf abscission in the fall~\cite{yang_2014_beyond,yang_2016_seasonal},
my results show a decline in leaf mass per area in the late growing season for all leaves followed by a slight increase at the end of the growing season for sunlit leaves.
The most likely explanation for this is inaccuracy in trait estimation, as evidenced by the extremely poor validation results for leaf mass per area for the phenological dataset (Figures~\ref{fig:project_validation_summary} and~\ref{fig:prospect_D_validation}).

Finally, optical traits revealed signatures of acute stress from insects, pathogens, and extreme environmental conditions.
In many cases, these effects agreed well with physiological expectations.
For instance, the significant negative effects of winter fleck, sucking and scale insects, and especially ozone damage on pine needles reported here match the earlier results of Di Vittorio (2009) for this dataset as well as the broader literature consensus on the damaging effects of ozone on plant physiology~\cite{lindroth_2010_impacts}. \nocite{divittorio_2009_pigment}
The same can be said for the adverse effects of Potato Virus Y on potato plants~\cite{scholthof_2011_top10}.
However, in several cases, the direction of these effects was counterintuitive.
Milkweed plants grown under elevated temperature and periodic drought stress~\cite{milkweed_data} showed the expected decline in leaf water content~\cite{penuelas_1994_reflectance,kramer_1995_water,cheng_2011_spectroscopic}, but showed a significant increase in pigment concentrations.
The higher concentrations of carotenoids under drought stress and anthocyanins under elevated temperature could reasonably be explained as photoprotective adaptations~\cite{young_1991_photoprotective,steyn_2002_anthocyanins,gould_2004_nature}.
The increased chlorophyll content is harder to explain, but similar increases in chlorophyll in drought stressed plants have been reported~\cite{vilfan_2016_fluspect}.
One possibility is that, because the chlorophyll is estimated on a leaf area basis, a reduction in leaf size and structure associated with declining water content could lead to an increase in apparent chlorophyll concentration, even if the mass-based concentration was constant or even slightly declined.
Regardless, the demonstrated ability of this study to not only detect but to analyze the physiological mechanisms of stress reinforces the value of leaf spectroscopy in both natural and agronomic settings.

\subsection{Patterns of trait correlation}

Optical traits vary non-randomly not only in response to environmental conditions and stress, but also with each other and with other ``invisible'' traits.
In Chapter 1, I found that the leaf economic spectrum---broadly, a multidimensional axis of trait variability defining a trade-off between productivity and resilience---generally held within plant functional types as well.
The findings in this chapter suggest that the leaf economic spectrum applies to variability among optical traits as well, as the second principal component of optical trait variability was characterized by a trade-off between productivity-related traits (pigments and water content) and structural traits (dry matter and mesophyll structure) (Figure~\ref{fig:prospect_pca}).
Both the explanatory power of the first two principal components (around 70\%) and their interpretation correspond remarkably well to a global analysis of species means from TRY~\cite{diaz_global_2016}.
Meanwhile, the third principal component of optical trait variability can be interpreted as an axis of stress, which can lead to reductions in leaf water content~\cite{penuelas_1994_reflectance,kramer_1995_water,cheng_2011_spectroscopic} and can promote higher investment in anthocyanins for their protective properties~\cite{gould_2004_nature}.
This is partially supported by my analysis of experimental treatments, water and anthocyanin contents did exhibit opposite responses to shade and warming (and to potato virus, though not significantly for anthocyanins) (Figure~\ref{fig:treatment_summary}).

The signature of the leaf economic spectrum was less clear among species means.
An economic trade-off between investment in pigments and structural molecules would imply negative correlations between these groups of traits,
but my results show that at the species level, pigments are weakly positively correlated with structural traits (Figure~\ref{fig:trait_correlations_species}).
Meanwhile, optical traits related to structure, as well as leaf water content, were strongly positively correlated with leaf C, cellulose, and lignin.
One explanation for these differences in correlations between pigments and structural molecules is that the former show relatively more plasticity whereas the latter are more phylogenetically conserved.
This is supported by the analysis of variance performed in this study (Figure~\ref{fig:within_vs_across}), as well as broader literature review of variability in leaf structure~\cite{poorter_2009_causes,onoda_physiological_2017}.
In addition, the modest positive interspecific correlation of leaf N with both structural molecules and pigments agrees with the large interspecific variability in leaf N allocation to structural and photosynthetic molecules~\cite{onoda_physiological_2017}.

Another key objective of this study was to investigate the ability of optical traits to predict other physiologically relevant traits that cannot be observed directly from spectra.
In many cases, multiple optical traits were significantly positively correlated with area-based leaf N, C, cellulose, lignin, $V_{c,\max}$, and $J_{\max}$,
which is not surprising given the extensive literature on empirical estimation of these traits from field and airborne spectroscopy~\cite{serbin_2011_leaf,serbin_spectroscopic_2014,asner_2015_quantifying,cavenderbares_2017_harnessing}
as well as the coordination between these traits in plant physiology and ecological strategy~\cite{kitajima_2003_increases,onoda_physiological_2017,croft_2017_chlorophyll}.
However, these correlations varied significantly between species, including for species within the same functional type (Figure~\ref{fig:trait_correlations}).
In part, this may be due to the unbalanced sampling in this analysis;
specifically, some species were sampled across a much wider range of conditions than others (particularly those grown under different experimental treatments), and correlations are likely to be more consistent where plants experience extreme conditions that lead to overall declines in leaf condition.
Patterns of trait covariance are also modulated by a number of environmental factors, such as dominant sources of limitation (e.g.\ light, water, or nutrients)~\cite{borgy_2017_plant} or strength of competitive effects~\cite{kunstler2016_competition} (see discussion in Chapter 1), which vary substantially across this dataset.
As such, identifying more precisely the drivers of variability in intraspecific trait correlations is an important future direction for this research.
